\cvsection{项目经历}
%----------------------------------------------------------------
%	内容
%----------------------------------------------------------------
\begin{cventries}

%---------------------------------------------------------
\cventry
{第一作者} % Job title
{真实图数据的分布外泛化} % Organization
{论文(投稿至ICML-24)} % Location
{2023.10-2024.02} % Date(s)
{
  \begin{cvitems} 
    \item {针对图OOD泛化领域的方法在手工数据集上表现良好但在真实数据中表现不佳的现状,提出更符合真实场景的图OOD泛化算法;}
    \item {不同于先前因果推断方法强调挖掘变量间的因果关系或独立性关系,本文从真实数据出发,提出发掘变量间的条件依赖性;}
    \item {根据变量间的相互条件依赖关系,构建变分推断的框架DEROG,结合EM算法对提出的DEROG框架进行优化;}
    \item {提出更符合真实场景的环境对齐函数和图主干的对比损失,提升DEROG对隐变量的推理能力;}
    \item {文章 “Improving Graph Out-of-distribution Generalization on Real-world Data” 投稿至机器学习顶会 (CCF-A) ICML-24。}
  \end{cvitems}
}

%---------------------------------------------------------
\cventry
{第二作者} % Job title
{异构图预训练上的对抗攻击方法} % Organization
{论文(投稿至KDD-24)} % Location
{2023.10-2024.02} % Date(s)
{
  \begin{cvitems} 
    \item {目前大量的异构图预训练模型被提出,但缺少针对其鲁棒性的研究,本文首次提出了针对异构图预训练的无监督攻击模型HGAC;}
    \item {HGAC考虑了攻击的隐蔽性和合理性,采用边缘重连的攻击策略,并使用节点聚类作为伪标签来指导攻击的优化;}
    \item {在节点分类、节点聚类、链接预测、可视化四种下游任务都证明了HGAC良好的攻击性能;}
    \item {文章 “Unsupervised Heterogeneous Graph Rewriting Attack via Node Clustering” 投稿至数据挖掘顶会 (CCF-A) KDD-24。}
  \end{cvitems}
}

%---------------------------------------------------------
\cventry
{第一作者} % Job title
{几何辅助的3D分子图扩散生成} % Organization
{论文} % Location
{2022.11-2023.08} % Date(s)
{
  \begin{cvitems} 
    \item {基于扩散生成模型,构建去噪扩散概率模型GFMDiff,生成创新且有效的3D分子结构;}
    \item {将扩散模型自身特点与原子间相互作用力本质相结合,构建出既能学习不同原子化学特性,又能学习任意原子间的相互作用强弱
    的全局注意力机制,构建创新的等变图网络作为去噪过程的内核;}
    \item {针对扩散模型在离散图数据中表现不佳的问题,创新性提出GFLoss,通过几何信息和理化性质引导模型生成真实且合理的图结构;}
    \item {文章“Geometric-facilitated denoising diffusion model for 3D molecule generation”已被人工智能顶会(CCF-A)AAAI-24 接收。}
  \end{cvitems}
}

%---------------------------------------------------------
\cventry
{共同第一作者 \& 实际撰稿人} % Job title
{扩散模型图生成综述} % Organization
{论文(投稿至TKDE)} % Location
{2023.02-至今} % Date(s)
{
  \begin{cvitems} 
    \item {系统性总结时下最先进的扩散生成模型在图生成任务上的应用;}
    \item {系统性阐述主流扩散模型范式,同时阐述并比较扩散图生成在计算化学,动作生成等领域的应用;}
    \item {讨论扩散模型在图生成任务中面临的不足与挑战,并对未来的研究方向提出展望。}
    \item {文章 “Diffusion-based graph generative methods” 投稿至数据挖掘顶刊 IEEE TRANSACTIONS ON KNOWLEDGE AND DATA ENGINEERING (中科院一区 / CCF-A) 。}
\end{cvitems}
}

%---------------------------------------------------------
% \cventry
% {队长} % Job title
% {\href{https://ogb.stanford.edu/neurips2022/results/}{OGB-LSC NeurlPS 2022}} % Organization
% {竞赛} % Location
% {2022.08-2022.11} % Date(s)
% {
%   \begin{cvitems} % Description(s) of tasks/responsibilities
%     \item {对超大规模分子数据(超300W)进行表征学习,预测目标分子能量隙;}
%     \item {构建HFAGNN模型,运用Hybrid模块,学习原子自身化学性质,同时运用Bessel方程提取2元组与3元组间的原子3D信息,将分子2D拓扑
%     信息与3D结构信息结合,在满足三维空间等变性的同时实现对分子性质的高效学习;}
%     \item {使用多卡并行实现高效运算,在较小规模参数量的情况下取得榜单第11名。}
%   \end{cvitems}
% }

%---------------------------------------------------------
\cventry
{第一作者} % Job title
{\href{https://www.sciencedirect.com/science/article/pii/S0020025523006497}{基于图对比学习的公平的标签感知推荐系统}} % Organization
{论文} % Location
{2022.05-2022.12} % Date(s)
{
  \begin{cvitems} % Description(s) of tasks/responsibilities
    \item {基于GNN和对比学习范式,构建TAGCL模型进一步提升基于标签感知的个性化推荐性能,同时显著提升推荐结果的公平性;}
    \item {分别构建<用户-标签>和<标签-商品>的子图,通过添加经过softmax归一化的随机扰动进行表征增强,用于建立对比学习任务;}
    \item {分别基于交叉熵损失,基于Tag负采样的二部图损失和改进知识图谱算法TransT,构建目标函数,分别用于推动生成更公平的推荐和提
    升二部图之间的一致性;}
    \item {实验显示模型推荐精度较标签感知推荐领域的SOTA模型至少提升5\%,同时在歧视性较强的数据集上显著提升结果的公平性;}
    \item {文章 “A fairness-aware graph contrastive learning recommender framework for social tagging systems” 发表于期刊INFORMATION SCIENCES (中科院一区(TOP) \& CCF-B)。}   
  \end{cvitems}
}

%---------------------------------------------------------
% \cventry
% {第二作者} % Job title
% {基于轻量化图结构的标签感知推荐系统} % Organization
% {论文} % Location
% {2022.03-2022.05} % Date(s)
% {
%   \begin{cvitems} % Description(s) of tasks/responsibilities
%     \item {为解决标签感知推荐系统中标签的模糊性和冗余性,构建TFGCF模型以实现对标签信息背后蕴含的隐式反馈的充分提取,模型较SOTA提 升至少5\%;}
%     \item {针对<用户-标签-商品>三元组分别基于<用户-标签>和<标签-商品>构建轻量化图结构。运用GCN进行信息传递和表征聚合,同时根据改 进的知识图谱算法TransRT提升标签信息的利用效率;}
%     \item {文章 “LFGCF: Light Folksonomy Graph Collaborative Filtering for Tag-aware Recommendation” 投稿至Expert systems 
%     with applications(中科院一区(TOP))。}
%   \end{cvitems}
% }

%---------------------------------------------------------
% \cventry
% {第一作者} % 职位名称
% {\href{https://www.frontiersin.org/articles/10.3389/fbioe.2022.827408/full}{基于深度强化学习的微分追逃策略研究}} % 组织
% {论文} % 位置
% {2021.09-2022.02} % 日期(s)
% {
%   \begin{cvitems} % 任务/职责描述
%     \item {对微分追逃博弈问题构建运动学方程,分别对追逃和逃逸策略进行运动学求解。}
%     \item {运用DQN和DDPG深度强化学习算法,实现了智能体的自动逃逸。}
%     \item {对Reward进行改进,进一步提升DQN模型训练智能体的逃逸成功率。}
%     \item {该文章 “Pursuit and Evasion Strategy of a Differential Game Based on Deep Reinforcement Learning” 发表于Frontiers in Bioengineering and Biotechnology (中科院二区 / JCR Q1)。}
%   \end{cvitems}
% }

%---------------------------------------------------------
% \cventry
% {组长} % Job title
% {电商用户商品价值评估与基于图神经网络的个性化推荐系统} % Organization
% {全国统计专业研究生案例大赛} % Location
% {2022.02-2022.03} % Date(s)
% {
%   \begin{cvitems} % Description(s) of tasks/responsibilities
%     \item {基于真实电商数据,实现商品从盈利能力、畅销水平和退货率等方面的价值评估,同时基于价值对用户群体进行有效划分,最后对用户
%     进行个性化推荐。}
%       % \item {基于一个真实电商平台交易数据,通过特征工程分别提取基于商品和用户的16维特征。}
%     \item {通过PCA降维与K-Means++聚类,实现对商品从盈利能力,畅销水平和退货率等多个维度的有效划分。}
%     \item {运用改进RFM模型,对用户特征做特征交叉,实现对用户从消费能力,下单频率和退货水平等多维度的有效划分。}
%     \item {基于LightGCN模型,创新性的提出了WideGCN模型,实现对基于商品和用户的16维特征这一旁信息的有效利用。推荐性能较SOTA模型有
%     2\%到5\%的提升。}
%     \item {该项目获全国三等奖}
%   \end{cvitems}
% }

%---------------------------------------------------------
% \cventry
% {组长} % Job title
% {空间抗干扰UWB定位算法} % Organization
% {全国研究生数学建模竞赛} % Location
% {2021.10} % Date(s)
% {
%   \begin{cvitems} % Description(s) of tasks/responsibilities
%     \item {构建方程实现分别基于3锚点和4锚点的空间UWB定位算法,实现较好的抗干扰性能。}
%     \item {利用SVM实现测量数据受干扰情况的准确判别。}
%     \item {利用上述定位算法与卡尔曼滤波实现物体在三维空间内的轨迹追踪。}
%     \item {获得第18届华为杯全国研究生数学建模竞赛三等奖。}
%   \end{cvitems}
% }

%---------------------------------------------------------
\end{cventries}
