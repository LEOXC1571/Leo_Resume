%!TEX TS-program = xelatex
%!TEX encoding = UTF-8 Unicode
%----------------------------------------------------------------
% CONFIGURATIONS
%----------------------------------------------------------------
% A4纸张大小默认情况下,'letterpaper'使用信纸
\documentclass[10pt, a4paper,AutoFakeBold]{awesome-cv}

\usepackage{hyperref}
% 使用geometry配置页边距
\geometry{left=1.0cm, top=.5cm, right=1.0cm, bottom=0.5cm, footskip=.5cm}

% 指定包含字体的位置
\fontdir[fonts/]

% 高光颜色
% Awesome Colors: awesome-emerald, awesome-skyblue, awesome-red, awesome-pink, awesome-orange
%                 awesome-nephritis, awesome-concrete, awesome-darknight
% \colorlet{awesome}{awesome-emerald}
% 如果要指定自己的颜色,请取消注释
\definecolor{awesome}{rgb}{0.2,0.4,0.7}

% 文本的颜色
% 如果要指定自己的颜色,请取消注释
% \definecolor{darktext}{HTML}{414141}
% \definecolor{text}{HTML}{333333}
% \definecolor{graytext}{HTML}{5D5D5D}
% \definecolor{lighttext}{HTML}{999999}

% Set false if you don't want to highlight section with awesome color
\setbool{acvSectionColorHighlight}{true}

% 如果要将社交信息分隔符从“|”更改为其他内容
\renewcommand{\acvHeaderSocialSep}{\quad\textbar\quad}


%----------------------------------------------------------------
%	如果要将社交信息分隔符从“|”更改为其他内容
%	如果不需要,请注释下面的任何一行
%----------------------------------------------------------------
% 可用选项: circle|rectangle,edge/noedge,left/right
\photo[circle,edge,right]{./pics/leo.jpg}
\name{徐}{璨}
\position{机器学习{\enskip\cdotp\enskip}数据挖掘}
% \address{英国伦敦贝克街221号}

\mobile{13812688705}
\email{leoxc1571@163.com}
\homepage{leoxc1571.github.io}
\github{leoxc1571}
% \linkedin{posquit0}
% \gitlab{gitlab-id}
% \stackoverflow{SO-id}{SO-name}
% \twitter{@twit}
% \skype{skype-id}
% \reddit{reddit-id}
% \medium{madium-id}
% \googlescholar{googlescholar-id}{name-to-display}
%% \firstname and \lastname will be used
% \googlescholar{googlescholar-id}{}
% \extrainfo{extra informations}

% \quote{``成为你想在世界上看到的改变。"}


%----------------------------------------------------------------
\begin{document}

% 打印具有以上个人信息的标题
% 提供可选参数以更改对齐方式(C: center, L: left, R: right)
\makecvheader

% 打印带有3个参数的页脚(<left>, <center>, <right>)
% 如果不需要,请将其中任何一项留空
\makecvfooter
  {\today}
  {徐璨~~~·~~~履历}
  {\thepage}


%----------------------------------------------------------------
%	CV 目录
%	每个部分是单独导入的,依次打开每个文件修改内容
%----------------------------------------------------------------
\cvsection{教育经历}
%	CONTENT
%----------------------------------------------------------------
\begin{educventries}
%---------------------------------------------------------
  \educventry
    {} % Degree
    {南京信息工程大学-应用统计学(本科)} % Institution
    {江苏南京} % Location
    {2017年9月-2021年6月} % Date(s)

%---------------------------------------------------------
  \educventry
    {以第一作者发表2区SCI论文一篇(深度强化学习),CCF-B会论文一篇(推荐系统)} % Degree
    {浙江工商大学-理学统计学(硕士)} % Institution
    {浙江杭州} % Location
    {2021年9月-2024年1月} % Date(s)

%---------------------------------------------------------
\end{educventries}

%----------------------------------------------------------------
%	SECTION TITLE
%----------------------------------------------------------------
\cvsection{技能}


%----------------------------------------------------------------
%	CONTENT
%----------------------------------------------------------------
\begin{cvskills}

%---------------------------------------------------------
  \cvskill
    {编程} % 类别
    {Python,SQL} % 技能

%---------------------------------------------------------
  \cvskill
    {技能} % 类别
    {PyTorch,Tensorflow,Recbole,DGL,Git,Linux,\LaTeX} % 技能

% %---------------------------------------------------------
%   \cvskill
%     {前端} % 类别
%     {Hugo, Redux, React, HTML5, LESS, SASS} % 技能

%---------------------------------------------------------
  % \cvskill
  %   {编程} % 类别
  %   {Node.js, Python, JAVA, OCaml, LaTeX} % 技能

%---------------------------------------------------------
  \cvskill
    {外语} % 类别
    {CET4(597),CET6(597)} % 技能

%---------------------------------------------------------
\end{cvskills}

% \cvsection{自我介绍}


%----------------------------------------------------------------
%	CONTENT
%----------------------------------------------------------------

\begin{cventries}

%---------------------------------------------------------
  \cvintro
      {
        \begin{cvitems} % Description(s) of tasks/responsibilities
          \item {熟练掌握Logistic回归、决策树、集成学习、聚类、降维等机器学习算法,熟悉并掌握各类GNN模型和推荐算法。}
          \item {熟练掌握PyTorch和Tensorflow深度学习框架。熟悉Recbole,DGL框架。具备较强的模型构建、复现、创新和调优的能力。}
        \end{cvitems}
      }
%---------------------------------------------------------

\end{cventries}
\cvsection{实习经历}
%----------------------------------------------------------------
%	内容
%----------------------------------------------------------------
\begin{cventries}
%---------------------------------------------------------
  \cventry
  {计算医药(药物性质学习、药物分子生成)} % Job title
  {之江实验室-图计算研究中心} % Organization
  {图算法工程师} % Location
  {2022年7月-2023年1月} % Date(s)
  {
    \begin{cvitems} % Description(s) of tasks/responsibilities
      \item {参与OGB-DDI榜单打榜,登顶全球第一;}
      \item {复现、部署图预训练模型,分子及构象生成模型}
      \item {实验显示模型较标签感知推荐领域的SOTA模型至少提升5\%。}
    \end{cvitems}
  }
%---------------------------------------------------------
  % \cventry
  %   {第一作者} % 职位名称
  %   {基于深度强化学习的微分追逃策略研究} % 组织
  %   {SCI论文} % 位置
  %   {2021年9月-2022年2月} % 日期(s)
  %   {
  %     \begin{cvitems} % 任务/职责描述
  %       \item {对微分追逃博弈问题构建运动学方程,分别对追逃和逃逸策略进行运动学求解。}
  %       \item {运用DQN和DDPG深度强化学习算法,实现了智能体的自动逃逸。}
  %       \item {对Reward进行改进,进一步提升DQN模型训练智能体的逃逸成功率。}
  %       \item {该文章 “Pursuit and Evasion Strategy of a Differential Game Based on Deep Reinforcement Learning” 发表于Frontiers in Bioengineering and Biotechnology。}
  %     \end{cvitems}
  %   }
%---------------------------------------------------------
%---------------------------------------------------------
\end{cventries}
\cvsection{项目经历}


%----------------------------------------------------------------
%	内容
%----------------------------------------------------------------
\begin{cventries}

%---------------------------------------------------------
  \cventry
    {第一作者} % 职位名称
    {基于深度强化学习的微分追逃策略研究} % 组织
    {SCI论文} % 位置
    {2021年9月-2022年2月} % 日期(s)
    {
      \begin{cvitems} % 任务/职责描述
        \item {对微分追逃博弈问题构建运动学方程,分别对追逃和逃逸策略进行运动学求解。}
        \item {运用DQN和DDPG深度强化学习算法,实现了智能体的自动逃逸。}
        \item {对Reward进行改进,进一步提升DQN模型训练智能体的逃逸成功率。}
        \item {该文章 “Pursuit and Evasion Strategy of a Differential Game Based on Deep Reinforcement Learning” 发表于Frontiers in Bioengineering and Biotechnology。}
      \end{cvitems}
    }
%---------------------------------------------------------
% \cventry
% {第一作者} % 职位名称
% {基于深度强化学习的微分追逃策略研究} % 组织
% {SCI论文} % 位置
% {2021年9月-2022年2月} % 日期(s)
% {
%   \begin{cvitems} % 任务/职责描述
%     \item {对微分追逃博弈问题构建运动学方程并求解。运用DQN和DDPG算法同时改进Reward机制,实现智能体逃逸策略学习和逃逸率提升。}
%     \item {该文章 “Pursuit and Evasion Strategy of a Differential Game Based on Deep Reinforcement Learning” 发表于Frontiers in Bioengineering and Biotechnology。}
%   \end{cvitems}
% }

%---------------------------------------------------------
  \cventry
    {组长} % Job title
    {空间抗干扰UWB定位算法} % Organization
    {全国研究生数学建模竞赛} % Location
    {2021年10月} % Date(s)
    {
      \begin{cvitems} % Description(s) of tasks/responsibilities
        \item {构建方程实现分别基于3锚点和4锚点的空间UWB定位算法,实现较好的抗干扰性能。}
        \item {利用SVM实现测量数据受干扰情况的准确判别。}
        \item {利用上述定位算法与卡尔曼滤波实现物体在三维空间内的轨迹追踪。}
        \item {获得第18届华为杯全国研究生数学建模竞赛三等奖。}
      \end{cvitems}
    }

% ---------------------------------------------------------
  \cventry
    {组长} % Job title
    {电商用户商品价值评估与基于图神经网络的个性化推荐系统} % Organization
    {全国统计专业研究生案例大赛} % Location
    {2022年2月-2022年3月} % Date(s)
    {
      \begin{cvitems} % Description(s) of tasks/responsibilities
        \item {基于真实电商数据,实现商品从盈利能力、畅销水平和退货率等方面的价值评估,同时基于价值对用户群体进行有效划分,最后
        对用户进行个性化推荐。}
        % \item {基于一个真实电商平台交易数据,通过特征工程分别提取基于商品和用户的16维特征。}
        \item {通过PCA降维与K-Means++聚类,实现对商品从盈利能力,畅销水平和退货率等多个维度的有效划分。}
        \item {运用改进RFM模型,对用户特征做特征交叉,实现对用户从消费能力,下单频率和退货水平等多维度的有效划分。}
        \item {基于LightGCN模型,创新性的提出了WideGCN模型,实现对基于商品和用户的16维特征这一旁信息的有效利用。推荐性能较SOTA
        模型有2\%到5\%的提升。}
        \item {该项目获}
      \end{cvitems}
    }

%---------------------------------------------------------
  \cventry
    {第二作者} % Job title
    {基于轻量化图结构的标签感知推荐系统} % Organization
    {SCI论文} % Location
    {2022年3月-2022年5月} % Date(s)
    {
      \begin{cvitems} % Description(s) of tasks/responsibilities
        \item {为解决标签感知推荐系统中标签的模糊性和冗余性,构建TFGCF模型以实现对标签信息背后蕴含的隐式反馈的充分提取,模型较SOTA提升至少5\%;}
        \item {针对<用户-标签-商品>三元组分别基于<用户-标签>和<标签-商品>构建轻量化图结构。运用GCN进行信息传递和表征聚合,同时根
        据改进的知识图谱算法TransRT提升标签信息的利用效率;}
        \item {该文章 “LFGCF: Light Folksonomy Graph Collaborative Filtering for Tag-aware Recommendation” 待发表。}
      \end{cvitems}
    }

%---------------------------------------------------------
  \cventry
    {第一作者} % Job title
    {基于图对比学习的标签感知推荐系统} % Organization
    {SCI论文} % Location
    {2022年5月至今} % Date(s)
    {
      \begin{cvitems} % Description(s) of tasks/responsibilities
        \item {基于GNN和对比学习范式,构建TAGCL模型进一步提升基于标签感知的个性化推荐性能;}
        \item {分别构建<用户-标签>和<标签-商品>的子图,通过添加经过softmax归一化的随机扰动进行表征增强,用于建立对比学习任务;}
        \item {分别基于交叉熵损失,基于Tag负采样的二部图损失和改进知识图谱算法TransT,构建目标函数;}
        \item {实验显示模型较标签感知推荐领域的SOTA模型提升较大。}
      \end{cvitems}
    }

%---------------------------------------------------------
\cventry
{参赛者} % Job title
{OGB-LSC NeurlPS 2022} % Organization
{竞赛} % Location
{2022年6月至今} % Date(s)
{
  \begin{cvitems} % Description(s) of tasks/responsibilities
    \item {运用图神经网络,代替密度泛函理论,预测分子属性;}
    \item {基于PCQM4M数据集,根据分子二维分子图结构,以平均绝对误差作为评价指标,预测HOMO-LUMO能量隙。}
    % \item {分别基于交叉熵损失,基于Tag负采样的二部图损失和改进知识图谱算法TransT,构建目标函数。}
    % \item {实验显示模型较标签感知推荐领域的SOTA模型提升较大。}
  \end{cvitems}
}

%---------------------------------------------------------
  % \cventry
  %   {自由职业渗透测试人员} % Job title
  %   {三星电子} % Organization
  %   {韩国} % Location
  %   {2013年9月、2011年3月-2011年10月} % Date(s)
  %   {
  %     \begin{cvitems} % Description(s) of tasks/responsibilities
  %       \item {对企业移动安全解决方案三星诺克斯进行渗透测试。}
  %       \item {对三星智能电视进行渗透测试。}
  %     \end{cvitems}
  %   }
%---------------------------------------------------------
\end{cventries}

% %----------------------------------------------------------------
%	SECTION TITLE
%----------------------------------------------------------------
\cvsection{课外活动}


%----------------------------------------------------------------
%	CONTENT
%----------------------------------------------------------------
\begin{cventries}

%---------------------------------------------------------
  \cventry
    {2013年核心成员兼总裁} % Affiliation/role
    {PoApper(POSTECH开发者网络)} % Organization/group
    {韩国浦项} % Location
    {2010年6月-2017年6月} % Date(s)
    {
      \begin{cvitems} % Description(s) of experience/contributions/knowledge
        \item {改革了以软件工程和校园内外网络建设为重点的社会。}
        \item {提出开展各种营销和网络活动,提高认识。}
      \end{cvitems}
    }

%---------------------------------------------------------
  \cventry
    {成员} % Affiliation/role
    {PLUS(POSTECH UNIX安全实验室)} % Organization/group
    {韩国浦项} % Location
    {2010年9月-2011年10月} % Date(s)
    {
      \begin{cvitems} % Description(s) of experience/contributions/knowledge
        \item {精通黑客和安全领域,尤其是基于UNIX的操作系统的内部攻击和多种攻击技术。}
        \item {参加了几次黑客竞赛,并获得了一个很好的奖项。}
        \item {作为POSTECH CERT的成员,定期对整个IT系统进行安全检查。}
        \item {由国家机构和公司委托进行渗透测试。}
      \end{cvitems}
    }

%---------------------------------------------------------
\end{cventries}

\cvsection{荣誉奖励}


%----------------------------------------------------------------
%	SUBSECTION TITLE
%----------------------------------------------------------------
% \cvsubsection{国际}


%----------------------------------------------------------------
%	CONTENT
%----------------------------------------------------------------
\begin{cvhonors}

%---------------------------------------------------------
  \cvhonor
    {2018年全国大学生英语竞赛(NECCS)} % Award
    {} % Event
    {全国一等奖} % Location
    {2018.5} % Date(s)

%---------------------------------------------------------
  \cvhonor
    {南京信息工程大学第十三届数学建模竞赛} % Award
    {} % Event
    {三等奖} % Location
    {2019.5} % Date(s)

%---------------------------------------------------------
\cvhonor
  {华为杯第十八届全国研究生数学建模竞赛} % Award
  {} % Event
  {全国三等奖} % Location
  {2021.10} % Date(s)

%---------------------------------------------------------
\cvhonor
  {浙江工商大学统计与数学学院研究生学业奖学金} % Award
  {} % Event
  {} % Location
  {2021.11} % Date(s)

%---------------------------------------------------------
%   \cvhonor
%     {特等奖} % Award
%     {DEFCON第22届CTF黑客大赛世界决赛} % Event
%     {美国拉斯维加斯} % Location
%     {2014} % Date(s)

% %---------------------------------------------------------
%   \cvhonor
%     {特等奖} % Award
%     {DEFCON第21届CTF黑客大赛世界决赛} % Event
%     {美国拉斯维加斯} % Location
%     {2013} % Date(s)

% %---------------------------------------------------------
%   \cvhonor
%     {特等奖} % Award
%     {DEFCON第19届CTF黑客大赛世界决赛} % Event
%     {美国拉斯维加斯} % Location
%     {2011} % Date(s)

%---------------------------------------------------------
\end{cvhonors}


%----------------------------------------------------------------
%	SUBSECTION TITLE
%----------------------------------------------------------------
% \cvsubsection{国内}


%----------------------------------------------------------------
%	CONTENT
%----------------------------------------------------------------
\begin{cvhonors}

%---------------------------------------------------------
%   \cvhonor
%     {第三名} % Award
%     {黑客大赛决赛} % Event
%     {中国深圳} % Location
%     {2015} % Date(s)

% %---------------------------------------------------------
%   \cvhonor
%     {银奖} % Award
%     {KISA HDCON黑客竞赛决赛} % Event
%     {中国深圳} % Location
%     {2017} % Date(s)

% %---------------------------------------------------------
%   \cvhonor
%     {银奖} % Award
%     {KISA HDCON黑客竞赛决赛} % Event
%     {中国深圳} % Location
%     {2013} % Date(s)

%---------------------------------------------------------
\end{cvhonors}

% %----------------------------------------------------------------
%	SECTION TITLE
%----------------------------------------------------------------
\cvsection{写作文档}


%----------------------------------------------------------------
%	CONTENT
%----------------------------------------------------------------
\begin{cventries}

%---------------------------------------------------------
  \cventry
    {创始人兼作家} % Role
    {初学者指南} % Title
    {Facebook页面} % Location
    {2015年1月至今} % Date(s)
    {
      \begin{cvitems} % Description(s)
        \item {为韩国的开发者起草关于IT技术和创业问题的每日新闻。}
      \end{cvitems}
    }

%---------------------------------------------------------
\end{cventries}

% \cvsection{项目委员}


%----------------------------------------------------------------
%	CONTENT
%----------------------------------------------------------------
\begin{cvhonors}

%---------------------------------------------------------
  \cvhonor
    {问题作者} % Position
    {2016年代码门黑客大赛世界决赛} % Committee
    {中国深圳} % Location
    {2016} % Date(s)

%---------------------------------------------------------
  \cvhonor
    {组织者兼联合主管} % Position
    {第一届POSTECH Hackathon} % Committee
    {中国深圳} % Location
    {2013} % Date(s)

%---------------------------------------------------------
\end{cvhonors}

%----------------------------------------------------------------
\end{document}
