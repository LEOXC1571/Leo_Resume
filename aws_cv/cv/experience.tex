\cvsection{项目经历}
%----------------------------------------------------------------
%	内容
%----------------------------------------------------------------
\begin{cventries}
%---------------------------------------------------------
  % \cventry
  %   {第一作者} % 职位名称
  %   {基于深度强化学习的微分追逃策略研究} % 组织
  %   {SCI论文} % 位置
  %   {2021年9月-2022年2月} % 日期(s)
  %   {
  %     \begin{cvitems} % 任务/职责描述
  %       \item {对微分追逃博弈问题构建运动学方程,分别对追逃和逃逸策略进行运动学求解。}
  %       \item {运用DQN和DDPG深度强化学习算法,实现了智能体的自动逃逸。}
  %       \item {对Reward进行改进,进一步提升DQN模型训练智能体的逃逸成功率。}
  %       \item {该文章 “Pursuit and Evasion Strategy of a Differential Game Based on Deep Reinforcement Learning” 发表于Frontiers in Bioengineering and Biotechnology。}
  %     \end{cvitems}
  %   }
%---------------------------------------------------------
% \cventry
% {第一作者} % 职位名称
% {基于深度强化学习的微分追逃策略研究} % 组织
% {SCI论文} % 位置
% {2021年9月-2022年2月} % 日期(s)
% {
%   \begin{cvitems} % 任务/职责描述
%     \item {对微分追逃博弈问题构建运动学方程并求解。运用DQN和DDPG算法同时改进Reward机制,实现智能体逃逸策略学习和逃逸率提升。}
%     \item {该文章 “Pursuit and Evasion Strategy of a Differential Game Based on Deep Reinforcement Learning” 发表于Frontiers in Bioengineering and Biotechnology。}
%   \end{cvitems}
% }
%---------------------------------------------------------
  % \cventry
  %   {组长} % Job title
  %   {空间抗干扰UWB定位算法} % Organization
  %   {全国研究生数学建模竞赛} % Location
  %   {2021年10月} % Date(s)
  %   {
  %     \begin{cvitems} % Description(s) of tasks/responsibilities
  %       \item {构建方程实现分别基于3锚点和4锚点的空间UWB定位算法,实现较好的抗干扰性能。}
  %       \item {利用SVM实现测量数据受干扰情况的准确判别。}
  %       \item {利用上述定位算法与卡尔曼滤波实现物体在三维空间内的轨迹追踪。}
  %       \item {获得第18届华为杯全国研究生数学建模竞赛三等奖。}
  %     \end{cvitems}
  %   }
% ---------------------------------------------------------
  % \cventry
  %   {组长} % Job title
  %   {电商用户商品价值评估与基于图神经网络的个性化推荐系统} % Organization
  %   {全国统计专业研究生案例大赛} % Location
  %   {2022年2月-2022年3月} % Date(s)
  %   {
  %     \begin{cvitems} % Description(s) of tasks/responsibilities
  %       \item {基于真实电商数据,实现商品从盈利能力、畅销水平和退货率等方面的价值评估,同时基于价值对用户群体进行有效划分,最后
  %       对用户进行个性化推荐。}
  %       % \item {基于一个真实电商平台交易数据,通过特征工程分别提取基于商品和用户的16维特征。}
  %       \item {通过PCA降维与K-Means++聚类,实现对商品从盈利能力,畅销水平和退货率等多个维度的有效划分。}
  %       \item {运用改进RFM模型,对用户特征做特征交叉,实现对用户从消费能力,下单频率和退货水平等多维度的有效划分。}
  %       \item {基于LightGCN模型,创新性的提出了WideGCN模型,实现对基于商品和用户的16维特征这一旁信息的有效利用。推荐性能较SOTA
  %       模型有2\%到5\%的提升。}
  %       \item {该项目获}
  %     \end{cvitems}
  %   }
%---------------------------------------------------------
  % \cventry
  %   {第二作者} % Job title
  %   {基于轻量化图结构的标签感知推荐系统} % Organization
  %   {SCI论文} % Location
  %   {2022年3月-2022年5月} % Date(s)
  %   {
  %     \begin{cvitems} % Description(s) of tasks/responsibilities
  %       \item {为解决标签感知推荐系统中标签的模糊性和冗余性,构建TFGCF模型以实现对标签信息背后蕴含的隐式反馈的充分提取,模型较SOTA提升至少5\%;}
  %       \item {针对<用户-标签-商品>三元组分别基于<用户-标签>和<标签-商品>构建轻量化图结构。运用GCN进行信息传递和表征聚合,同时根
  %       据改进的知识图谱算法TransRT提升标签信息的利用效率;}
  %       \item {该文章 “LFGCF: Light Folksonomy Graph Collaborative Filtering for Tag-aware Recommendation” 待发表。}
  %     \end{cvitems}
  %   }
%---------------------------------------------------------
\cventry
  {第一作者} % Job title
  {\href{https://www.sciencedirect.com/science/article/pii/S0020025523006497}{基于图对比学习的公平的标签感知推荐系统}} % Organization
  {JCR-Q1(TOP)\&CCF-B} % Location
  {2022年5月-2022年12月} % Date(s)
  {
    \begin{cvitems} % Description(s) of tasks/responsibilities
      \item {基于GNN和对比学习范式,构建TAGCL模型进一步提升基于标签感知的个性化推荐性能,同时显著提升推荐结果的公平性;}
      \item {分别构建<用户-标签>和<标签-商品>的子图,通过添加经过softmax归一化的随机扰动进行表征增强,用于建立对比学习任务;}
      \item {分别基于交叉熵损失,基于Tag负采样的二部图损失和改进知识图谱算法TransT,构建目标函数,分别用于推动生成更公平的推荐和提
      升二部图之间的一致性;}
      \item {实验显示模型推荐精度较标签感知推荐领域的SOTA模型至少提升5\%,同时在歧视性较强的数据集上显著提升结果的公平性。}
    \end{cvitems}
  }
%---------------------------------------------------------
\cventry
  {队长} % Job title
  {OGB-LSC NeurlPS 2022} % Organization
  {竞赛} % Location
  {2022年8月-2022年11月} % Date(s)
  {
    \begin{cvitems} % Description(s) of tasks/responsibilities
      \item {对超大规模分子数据(超200W)进行表针学习,预测目标分子能量隙;}
      \item {构建HFAGNN模型,运用Hybrid模块,学习原子自身化学性质,同时运用Bessel方程提取2元组与3元组间的原子3D信息,将分子2D拓扑
      信息与3D结构信息结合,在满足三维空间等变性的同时实现对分子性质的高效学习;}
      \item {使用多卡并行实现高效运算,在较小规模参数量的情况下取得榜单第11名。}
    \end{cvitems}
  }
%---------------------------------------------------------

\

\cventry
  {负责人} % Job title
  {基于扩散模型的3D分子生成} % Organization
  {论文(撰写中)} % Location
  {2022年11月-至今} % Date(s)
  {
    \begin{cvitems} 
      \item {基于扩散生成模型,构建去噪扩散概率模型,生成创新且有效的3D分子结构;}
      \item {将扩散模型自身特点与原子间相互作用力本质相结合,构建出既能学习不同原子化学特性,又能学习任意原子间的相互作用强弱的全局
      注意力机制,构建创新的等变图网络作为去噪过程的内核;}
      \item {改进原有采样策略,引导模型在生成大分子过程中做出更多探索,提升生成分子的创新性。}
    \end{cvitems}
  }

\cventry
  {组长} % Job title
  {电商用户商品价值评估与基于图神经网络的个性化推荐系统} % Organization
  {全国统计专业研究生案例大赛} % Location
  {2022年2月-2022年3月} % Date(s)
  {
    \begin{cvitems} % Description(s) of tasks/responsibilities
      \item {基于真实电商数据,实现商品从盈利能力、畅销水平和退货率等方面的价值评估,同时基于价值对用户群体进行有效划分,最后对用户
      进行个性化推荐。}
      % \item {基于一个真实电商平台交易数据,通过特征工程分别提取基于商品和用户的16维特征。}
      \item {通过PCA降维与K-Means++聚类,实现对商品从盈利能力,畅销水平和退货率等多个维度的有效划分。}
      \item {运用改进RFM模型,对用户特征做特征交叉,实现对用户从消费能力,下单频率和退货水平等多维度的有效划分。}
      \item {基于LightGCN模型,创新性的提出了WideGCN模型,实现对基于商品和用户的16维特征这一旁信息的有效利用。推荐性能较SOTA模型有
      2\%到5\%的提升。}
      \item {该项目获全国三等奖}
    \end{cvitems}
  }
  
%---------------------------------------------------------
  % \cventry
  %   {自由职业渗透测试人员} % Job title
  %   {三星电子} % Organization
  %   {韩国} % Location
  %   {2013年9月、2011年3月-2011年10月} % Date(s)
  %   {
  %     \begin{cvitems} % Description(s) of tasks/responsibilities
  %       \item {对企业移动安全解决方案三星诺克斯进行渗透测试。}
  %       \item {对三星智能电视进行渗透测试。}
  %     \end{cvitems}
  %   }
%---------------------------------------------------------
\end{cventries}
