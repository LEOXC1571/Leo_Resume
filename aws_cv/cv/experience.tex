\cvsection{项目经历}
%----------------------------------------------------------------
%	内容
%----------------------------------------------------------------
\begin{cventries}

%---------------------------------------------------------
\cventry
{第一作者} % Job title
{\href{https://www.sciencedirect.com/science/article/pii/S0020025523006497}{基于图对比学习的公平的标签感知推荐系统}} % Organization
{论文} % Location
{2022年5月-2022年12月} % Date(s)
{
  \begin{cvitems} % Description(s) of tasks/responsibilities
    \item {基于GNN和对比学习范式,构建TAGCL模型进一步提升基于标签感知的个性化推荐性能,同时显著提升推荐结果的公平性;}
    \item {分别构建<用户-标签>和<标签-商品>的子图,通过添加经过softmax归一化的随机扰动进行表征增强,用于建立对比学习任务;}
    \item {分别基于交叉熵损失,基于Tag负采样的二部图损失和改进知识图谱算法TransT,构建目标函数,分别用于推动生成更公平的推荐和提
    升二部图之间的一致性;}
    \item {实验显示模型推荐精度较标签感知推荐领域的SOTA模型至少提升5\%,同时在歧视性较强的数据集上显著提升结果的公平性;}
    \item {文章 “A fairness-aware graph contrastive learning recommender framework for social tagging systems” 发表于期刊INFORMATION SCIENCES (中科院一区(TOP) \& CCF-B)。}   
  \end{cvitems}
}
  
%---------------------------------------------------------
\cventry
{第一作者} % Job title
{几何辅助的3D分子图扩散生成} % Organization
{论文(投稿至AAAI-24)} % Location
{2022年11月-至今} % Date(s)
{
  \begin{cvitems} 
    \item {基于扩散生成模型,构建去噪扩散概率模型,生成创新且有效的3D分子结构;}
    \item {将扩散模型自身特点与原子间相互作用力本质相结合,构建出既能学习不同原子化学特性,又能学习任意原子间的相互作用强弱的全局
    注意力机制,构建创新的等变图网络作为去噪过程的内核;}
    \item {改进原有采样策略,引导模型在生成大分子过程中做出更多探索,提升生成分子的创新性;}
    \item {文章 “Geometric-facilitated denoising diffusion model for 3D molecule generation” 投稿至人工智能顶会 (CCF-A) AAAI-24。}
  \end{cvitems}
}

%---------------------------------------------------------
\cventry
{共同第一作者 \& 实际撰稿人} % Job title
{扩散模型图生成综述} % Organization
{论文(投稿至TKDE)} % Location
{2023年2月-至今} % Date(s)
{
  \begin{cvitems} 
    \item {系统性总结时下最先进的扩散生成模型在图生成任务上的应用;}
    \item {系统性阐述主流扩散模型范式,同时阐述并比较扩散图生成在计算化学,动作生成等领域的应用;}
    \item {讨论扩散模型在图生成任务中面临的不足与挑战,并对未来的研究方向提出展望。}
    \item {文章 “Diffusion-based graph generative methods” 投稿至数据挖掘顶刊 IEEE TRANSACTIONS ON KNOWLEDGE AND DATA ENGINEERING (中科院一区 / CCF-A) 。}
\end{cvitems}
}

%---------------------------------------------------------
\cventry
{队长} % Job title
{\href{https://ogb.stanford.edu/neurips2022/results/}{OGB-LSC NeurlPS 2022}} % Organization
{竞赛} % Location
{2022年8月-2022年11月} % Date(s)
{
  \begin{cvitems} % Description(s) of tasks/responsibilities
    \item {对超大规模分子数据(超300W)进行表征学习,预测目标分子能量隙;}
    \item {构建HFAGNN模型,运用Hybrid模块,学习原子自身化学性质,同时运用Bessel方程提取2元组与3元组间的原子3D信息,将分子2D拓扑
    信息与3D结构信息结合,在满足三维空间等变性的同时实现对分子性质的高效学习;}
    \item {使用多卡并行实现高效运算,在较小规模参数量的情况下取得榜单第11名。}
  \end{cvitems}
}

%---------------------------------------------------------
\cventry
{第一作者} % 职位名称
{\href{https://www.frontiersin.org/articles/10.3389/fbioe.2022.827408/full}{基于深度强化学习的微分追逃策略研究}} % 组织
{论文} % 位置
{2021年9月-2022年2月} % 日期(s)
{
  \begin{cvitems} % 任务/职责描述
    \item {对微分追逃博弈问题构建运动学方程,分别对追逃和逃逸策略进行运动学求解。}
    \item {运用DQN和DDPG深度强化学习算法,实现了智能体的自动逃逸。}
    \item {对Reward进行改进,进一步提升DQN模型训练智能体的逃逸成功率。}
    \item {该文章 “Pursuit and Evasion Strategy of a Differential Game Based on Deep Reinforcement Learning” 发表于Frontiers in Bioengineering and Biotechnology (中科院二区 / JCR Q1)。}
  \end{cvitems}
}

%---------------------------------------------------------
\cventry
{第二作者} % Job title
{基于轻量化图结构的标签感知推荐系统} % Organization
{论文} % Location
{2022年3月-2022年5月} % Date(s)
{
  \begin{cvitems} % Description(s) of tasks/responsibilities
    \item {为解决标签感知推荐系统中标签的模糊性和冗余性,构建TFGCF模型以实现对标签信息背后蕴含的隐式反馈的充分提取,模型较SOTA提 升至少5\%;}
    \item {针对<用户-标签-商品>三元组分别基于<用户-标签>和<标签-商品>构建轻量化图结构。运用GCN进行信息传递和表征聚合,同时根据改 进的知识图谱算法TransRT提升标签信息的利用效率;}
    \item {文章 “LFGCF: Light Folksonomy Graph Collaborative Filtering for Tag-aware Recommendation” 投稿至Expert systems 
    with applications(中科院一区(TOP))。}
  \end{cvitems}
}

%---------------------------------------------------------
\cventry
{组长} % Job title
{电商用户商品价值评估与基于图神经网络的个性化推荐系统} % Organization
{全国统计专业研究生案例大赛} % Location
{2022年2月-2022年3月} % Date(s)
{
  \begin{cvitems} % Description(s) of tasks/responsibilities
    \item {基于真实电商数据,实现商品从盈利能力、畅销水平和退货率等方面的价值评估,同时基于价值对用户群体进行有效划分,最后对用户
    进行个性化推荐。}
      % \item {基于一个真实电商平台交易数据,通过特征工程分别提取基于商品和用户的16维特征。}
    \item {通过PCA降维与K-Means++聚类,实现对商品从盈利能力,畅销水平和退货率等多个维度的有效划分。}
    \item {运用改进RFM模型,对用户特征做特征交叉,实现对用户从消费能力,下单频率和退货水平等多维度的有效划分。}
    \item {基于LightGCN模型,创新性的提出了WideGCN模型,实现对基于商品和用户的16维特征这一旁信息的有效利用。推荐性能较SOTA模型有
    2\%到5\%的提升。}
    \item {该项目获全国三等奖}
  \end{cvitems}
}

%---------------------------------------------------------
\cventry
{组长} % Job title
{空间抗干扰UWB定位算法} % Organization
{全国研究生数学建模竞赛} % Location
{2021年10月} % Date(s)
{
  \begin{cvitems} % Description(s) of tasks/responsibilities
    \item {构建方程实现分别基于3锚点和4锚点的空间UWB定位算法,实现较好的抗干扰性能。}
    \item {利用SVM实现测量数据受干扰情况的准确判别。}
    \item {利用上述定位算法与卡尔曼滤波实现物体在三维空间内的轨迹追踪。}
    \item {获得第18届华为杯全国研究生数学建模竞赛三等奖。}
  \end{cvitems}
}

%---------------------------------------------------------
\end{cventries}
