\cvsection{实习经历}
%----------------------------------------------------------------
%	内容
%----------------------------------------------------------------
\begin{cventries}
%---------------------------------------------------------
  \cventry
  {智能计算计算医药(大规模图预训练,分子性质预测,Diffusion-based分子生成)} % Job title
  {之江实验室-图计算研究中心} % Organization
  {图算法工程师} % Location
  {2022年7月-至今} % Date(s)
  {
    \begin{cvitems} % Description(s) of tasks/responsibilities
      \item {负责OGB-LSC NeurIPS22竞赛,构建模型实现超大规模分子图性质学习,取得全球第11名;}
      \item {复现、部署大规模图预训练模型及分子构象生成模型;}
      \item {负责Diffusion-based生成模型研究,撰写Diffusion-based图生成领域综述;}
      \item {创造性提出基于全局注意力的E(n)等变图网络,结合原子自身性质与分子间相互作用,实现时下最优的3D分子学习性能。同时将此图
      网络与Diffusion模型结合,实现时下最优的3D分子生成性能,相关文章正在撰写中。}
    \end{cvitems}
  }
%---------------------------------------------------------
  % \cventry
  %   {第一作者} % 职位名称
  %   {基于深度强化学习的微分追逃策略研究} % 组织
  %   {SCI论文} % 位置
  %   {2021年9月-2022年2月} % 日期(s)
  %   {
  %     \begin{cvitems} % 任务/职责描述
  %       \item {对微分追逃博弈问题构建运动学方程,分别对追逃和逃逸策略进行运动学求解。}
  %       \item {运用DQN和DDPG深度强化学习算法,实现了智能体的自动逃逸。}
  %       \item {对Reward进行改进,进一步提升DQN模型训练智能体的逃逸成功率。}
  %       \item {该文章 “Pursuit and Evasion Strategy of a Differential Game Based on Deep Reinforcement Learning” 发表于Frontiers in Bioengineering and Biotechnology。}
  %     \end{cvitems}
  %   }
%---------------------------------------------------------
%---------------------------------------------------------
\end{cventries}